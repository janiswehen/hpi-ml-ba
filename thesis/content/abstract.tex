\null\vfil
\begin{otherlanguage}{english}
\begin{center}\textsf{\textbf{\abstractname}}\end{center}

\noindent Deep learning has brought significant advancements in medical imaging,
with the U-Net model being a popular choice for image segmentation.
However, the 3D U-Net often needs a lot of computer memory, especially when working with large images.
This thesis focuses on modifying the U-Net design to use less memory without losing much accuracy.
We study two main designs: the patch-wise U-Net and the patch-wise U-Net cascade.
Our tests on the Brain Tumor and Liver datasets show that while the regular 3D U-Net is best in terms of accuracy it is
not feasible to use it for images with a large voxel count since it uses up too mutch GPU memory.
The patch-wise U-Net cascade shows to be a promising alternative for larger images that is less demanding on GPU memory resources.
This work offers a path forward for using advanced image processing in situations where computer resources are limited.
\end{otherlanguage}
\vfil\null
