\null\vfil
\begin{otherlanguage}{ngerman}
\begin{center}\textsf{\textbf{\abstractname}}\end{center}

\noindent Mit hilfe von Deep Learning sind im Bereich der medizinischen Bildgebung bedeutende Fortschritte gelungen,
wobei das U-Net Modell als eine der bevorzugten Methoden für die Bildsegmentierung bekannt ist.
Das $3$D U-Net weist jedoch oft einen hohen GPU-Speicherbedarf auf, besonders bei der Verarbeitung großer $3$D-Bilddaten.
Diese Studie zielt darauf ab, die U-Net Struktur so anzupassen, dass sie weniger Speicher verbraucht,
ohne erheblich an Genauigkeit einzubüßen. Hierbei stehen zwei Modellvarianten des U-Nets im Fokus:
das `patch-wise U-Net' und die `patch-wise U-Net cascade'. Vergleichstests anhand der Datensätze `Brain Tumor'
und `Liver' haben ergeben, dass das herkömmliche $3$D U-Net in puncto Genauigkeit führend ist.
Allerdings erweist es sich bei Bildern mit hoher Voxelanzahl als nicht umsetzbar aufgrund des exzessiven GPU-Speicherverbrauchs.
Das `patch-wise U-Net cascade' stellt sich als eine vielversprechende Alternative heraus, speziell für hochaufgelöste Bilder,
da es effizienter in Bezug auf den GPU-Verbrauch ist.
Dieser Ansatz ermöglicht fortschrittliche Bildsegmentierungen auch bei eingeschränkten Rechenkapazitäten.
\end{otherlanguage}
\vfil\null
